\documentclass[]{article}
\usepackage{lmodern}
\usepackage{amssymb,amsmath}
\usepackage{ifxetex,ifluatex}
\usepackage{fixltx2e} % provides \textsubscript
\ifnum 0\ifxetex 1\fi\ifluatex 1\fi=0 % if pdftex
  \usepackage[T1]{fontenc}
  \usepackage[utf8]{inputenc}
\else % if luatex or xelatex
  \ifxetex
    \usepackage{mathspec}
  \else
    \usepackage{fontspec}
  \fi
  \defaultfontfeatures{Ligatures=TeX,Scale=MatchLowercase}
  \newcommand{\euro}{€}
\fi
% use upquote if available, for straight quotes in verbatim environments
\IfFileExists{upquote.sty}{\usepackage{upquote}}{}
% use microtype if available
\IfFileExists{microtype.sty}{%
\usepackage{microtype}
\UseMicrotypeSet[protrusion]{basicmath} % disable protrusion for tt fonts
}{}
\usepackage[margin=1in]{geometry}
\usepackage{hyperref}
\PassOptionsToPackage{usenames,dvipsnames}{color} % color is loaded by hyperref
\hypersetup{unicode=true,
            pdftitle={UAV Drone autopilot, computer vision and image processing},
            pdfborder={0 0 0},
            breaklinks=true}
\urlstyle{same}  % don't use monospace font for urls
\usepackage{color}
\usepackage{fancyvrb}
\newcommand{\VerbBar}{|}
\newcommand{\VERB}{\Verb[commandchars=\\\{\}]}
\DefineVerbatimEnvironment{Highlighting}{Verbatim}{commandchars=\\\{\}}
% Add ',fontsize=\small' for more characters per line
\usepackage{framed}
\definecolor{shadecolor}{RGB}{248,248,248}
\newenvironment{Shaded}{\begin{snugshade}}{\end{snugshade}}
\newcommand{\KeywordTok}[1]{\textcolor[rgb]{0.13,0.29,0.53}{\textbf{{#1}}}}
\newcommand{\DataTypeTok}[1]{\textcolor[rgb]{0.13,0.29,0.53}{{#1}}}
\newcommand{\DecValTok}[1]{\textcolor[rgb]{0.00,0.00,0.81}{{#1}}}
\newcommand{\BaseNTok}[1]{\textcolor[rgb]{0.00,0.00,0.81}{{#1}}}
\newcommand{\FloatTok}[1]{\textcolor[rgb]{0.00,0.00,0.81}{{#1}}}
\newcommand{\ConstantTok}[1]{\textcolor[rgb]{0.00,0.00,0.00}{{#1}}}
\newcommand{\CharTok}[1]{\textcolor[rgb]{0.31,0.60,0.02}{{#1}}}
\newcommand{\SpecialCharTok}[1]{\textcolor[rgb]{0.00,0.00,0.00}{{#1}}}
\newcommand{\StringTok}[1]{\textcolor[rgb]{0.31,0.60,0.02}{{#1}}}
\newcommand{\VerbatimStringTok}[1]{\textcolor[rgb]{0.31,0.60,0.02}{{#1}}}
\newcommand{\SpecialStringTok}[1]{\textcolor[rgb]{0.31,0.60,0.02}{{#1}}}
\newcommand{\ImportTok}[1]{{#1}}
\newcommand{\CommentTok}[1]{\textcolor[rgb]{0.56,0.35,0.01}{\textit{{#1}}}}
\newcommand{\DocumentationTok}[1]{\textcolor[rgb]{0.56,0.35,0.01}{\textbf{\textit{{#1}}}}}
\newcommand{\AnnotationTok}[1]{\textcolor[rgb]{0.56,0.35,0.01}{\textbf{\textit{{#1}}}}}
\newcommand{\CommentVarTok}[1]{\textcolor[rgb]{0.56,0.35,0.01}{\textbf{\textit{{#1}}}}}
\newcommand{\OtherTok}[1]{\textcolor[rgb]{0.56,0.35,0.01}{{#1}}}
\newcommand{\FunctionTok}[1]{\textcolor[rgb]{0.00,0.00,0.00}{{#1}}}
\newcommand{\VariableTok}[1]{\textcolor[rgb]{0.00,0.00,0.00}{{#1}}}
\newcommand{\ControlFlowTok}[1]{\textcolor[rgb]{0.13,0.29,0.53}{\textbf{{#1}}}}
\newcommand{\OperatorTok}[1]{\textcolor[rgb]{0.81,0.36,0.00}{\textbf{{#1}}}}
\newcommand{\BuiltInTok}[1]{{#1}}
\newcommand{\ExtensionTok}[1]{{#1}}
\newcommand{\PreprocessorTok}[1]{\textcolor[rgb]{0.56,0.35,0.01}{\textit{{#1}}}}
\newcommand{\AttributeTok}[1]{\textcolor[rgb]{0.77,0.63,0.00}{{#1}}}
\newcommand{\RegionMarkerTok}[1]{{#1}}
\newcommand{\InformationTok}[1]{\textcolor[rgb]{0.56,0.35,0.01}{\textbf{\textit{{#1}}}}}
\newcommand{\WarningTok}[1]{\textcolor[rgb]{0.56,0.35,0.01}{\textbf{\textit{{#1}}}}}
\newcommand{\AlertTok}[1]{\textcolor[rgb]{0.94,0.16,0.16}{{#1}}}
\newcommand{\ErrorTok}[1]{\textcolor[rgb]{0.64,0.00,0.00}{\textbf{{#1}}}}
\newcommand{\NormalTok}[1]{{#1}}
\usepackage{graphicx,grffile}
\makeatletter
\def\maxwidth{\ifdim\Gin@nat@width>\linewidth\linewidth\else\Gin@nat@width\fi}
\def\maxheight{\ifdim\Gin@nat@height>\textheight\textheight\else\Gin@nat@height\fi}
\makeatother
% Scale images if necessary, so that they will not overflow the page
% margins by default, and it is still possible to overwrite the defaults
% using explicit options in \includegraphics[width, height, ...]{}
\setkeys{Gin}{width=\maxwidth,height=\maxheight,keepaspectratio}
\setlength{\parindent}{0pt}
\setlength{\parskip}{6pt plus 2pt minus 1pt}
\setlength{\emergencystretch}{3em}  % prevent overfull lines
\providecommand{\tightlist}{%
  \setlength{\itemsep}{0pt}\setlength{\parskip}{0pt}}
\setcounter{secnumdepth}{0}

%%% Use protect on footnotes to avoid problems with footnotes in titles
\let\rmarkdownfootnote\footnote%
\def\footnote{\protect\rmarkdownfootnote}

%%% Change title format to be more compact
\usepackage{titling}

% Create subtitle command for use in maketitle
\newcommand{\subtitle}[1]{
  \posttitle{
    \begin{center}\large#1\end{center}
    }
}

\setlength{\droptitle}{-2em}
  \title{UAV Drone autopilot, computer vision and image processing}
  \pretitle{\vspace{\droptitle}\centering\huge}
  \posttitle{\par}
  \author{}
  \preauthor{}\postauthor{}
  \date{}
  \predate{}\postdate{}


% Redefines (sub)paragraphs to behave more like sections
\ifx\paragraph\undefined\else
\let\oldparagraph\paragraph
\renewcommand{\paragraph}[1]{\oldparagraph{#1}\mbox{}}
\fi
\ifx\subparagraph\undefined\else
\let\oldsubparagraph\subparagraph
\renewcommand{\subparagraph}[1]{\oldsubparagraph{#1}\mbox{}}
\fi


\begin{document}
\maketitle

\subsection{Course Informations}\label{course-informations}

\begin{itemize}
\tightlist
\item
  No textbook
\item
  Spring 2018
\item
  Lecture: K.W. Chen in \href{mailto:CS@NCTU}{\nolinkurl{CS@NCTU}}
\end{itemize}

\subsubsection{Lab1. Introduction to Open CV and Installation Mar 01,
2018}\label{lab1.-introduction-to-open-cv-and-installation-mar-01-2018}

Lab1.pdf

\paragraph{Install Open CV}\label{install-open-cv}

In the Ubuntu 16.04 64bit Required dependencies are the followings:

\begin{enumerate}
\def\labelenumi{\arabic{enumi}.}
\tightlist
\item
  Git
\item
  Python 2.6 or later and Numpy 1.5 or later with developer packages
  (python-dev, python-numpy)
\item
  CMake 2.6 or higher
\item
  GCC 4.4.x or later
\end{enumerate}

\begin{Shaded}
\begin{Highlighting}[]
\KeywordTok{sudo} \NormalTok{apt-get install libopencv-dev python-opencv}
\end{Highlighting}
\end{Shaded}

\subparagraph{Once installed}\label{once-installed}

\begin{Shaded}
\begin{Highlighting}[]
\KeywordTok{pkg-config} \NormalTok{--modversion opencv}
\end{Highlighting}
\end{Shaded}

If the version is shown, then we're good to rock with opencv.

\begin{itemize}
\tightlist
\item
  Build your opencv\_file.cpp with CMake or g++ standards
\end{itemize}

\begin{enumerate}
\def\labelenumi{\arabic{enumi}.}
\item
  With CMake (Don't forget to add CMakeLists.txt)

\begin{verbatim}
cmake_minimum_required(VERSION 2.8)
project( <project_name> )
find_package( OpenCV REQUIRED )
add_executable( <project_name> <project_name>.cpp )
target_link_libraries( <project_name> ${OpenCV_LIBS} )
\end{verbatim}
\item
  With g++ and flags for opencv libraries

\begin{Shaded}
\begin{Highlighting}[]
\KeywordTok{g++} \NormalTok{lab1-2.cpp }\KeywordTok{`pkg-config} \NormalTok{--cflags --libs opencv}\KeywordTok{`}
\end{Highlighting}
\end{Shaded}
\end{enumerate}

\subparagraph{Course Content}\label{course-content}

CourseWeek1.pdf

\subsubsection{Lab2. Histogram Equalization and Laplacian
sharpening}\label{lab2.-histogram-equalization-and-laplacian-sharpening}

Lab2.pdf

Note: The default image matrix is three-channel type, we have to do the
gray scale conversion, only ⅓ of the image will be processed.

\begin{Shaded}
\begin{Highlighting}[]
\NormalTok{Mat input_img = imread(argv[}\DecValTok{1}\NormalTok{]);}
\CommentTok{//since the bgr channel is used for default action, then the BGR 3 channel image must be converted to GREY channel}

\NormalTok{cvtColor(input_img, input_img, CV_BGR2GRAY);}
\NormalTok{Mat output_img = input_img.clone();}
\NormalTok{histogram_equal(input_img, output_img);}
\end{Highlighting}
\end{Shaded}

Histogram Equalization Function:

\begin{Shaded}
\begin{Highlighting}[]
\NormalTok{oid histogram_equal(Mat& input, Mat& output)}
\NormalTok{\{}
    \NormalTok{vector<}\DataTypeTok{int}\NormalTok{> hash_distribution;}
    \NormalTok{vector<}\DataTypeTok{double}\NormalTok{> intensity_cdf;}
    \NormalTok{hash_distribution.resize(}\DecValTok{256}\NormalTok{);}
    \NormalTok{intensity_cdf.resize(}\DecValTok{256}\NormalTok{);}
    \KeywordTok{for}\NormalTok{(}\DataTypeTok{int} \NormalTok{i=}\DecValTok{0}\NormalTok{;i<input.rows;i++)}
    \NormalTok{\{}
        \KeywordTok{for}\NormalTok{(}\DataTypeTok{int} \NormalTok{j=}\DecValTok{0}\NormalTok{;j<input.cols;j++)}
        \NormalTok{\{}
            \NormalTok{hash_distribution[(}\DataTypeTok{int}\NormalTok{) input.at<}\DataTypeTok{uchar}\NormalTok{>(i,j)]++;}
        \NormalTok{\}}
    \NormalTok{\}}

    \CommentTok{//search the maxium value}
    \DataTypeTok{int} \NormalTok{max_value = }\DecValTok{0}\NormalTok{, cnt=}\DecValTok{0}\NormalTok{;}
    \DataTypeTok{double} \NormalTok{cumulative_cnt = }\FloatTok{0.0f}\NormalTok{;}
    \KeywordTok{for}\NormalTok{(}\DataTypeTok{int} \NormalTok{i=}\DecValTok{0}\NormalTok{;i<hash_distribution.size();i++)}
    \NormalTok{\{}
        \KeywordTok{if}\NormalTok{(hash_distribution[i]!=}\DecValTok{0}\NormalTok{)}
        \NormalTok{\{}
            \NormalTok{max_value = max(max_value, i);}
            \NormalTok{cumulative_cnt += (}\DataTypeTok{double}\NormalTok{) hash_distribution[i] / (}\DataTypeTok{double}\NormalTok{)(input.rows * input.cols);}
            \NormalTok{cnt += hash_distribution[i];}
            \NormalTok{intensity_cdf[i] = cumulative_cnt;}
        \NormalTok{\}}
    \NormalTok{\}}
    \KeywordTok{for}\NormalTok{(}\DataTypeTok{int} \NormalTok{i=}\DecValTok{0}\NormalTok{;i<input.rows;i++)}
    \NormalTok{\{}
        \KeywordTok{for}\NormalTok{(}\DataTypeTok{int} \NormalTok{j=}\DecValTok{0}\NormalTok{;j<input.cols;j++)}
        \NormalTok{\{}
            \NormalTok{output.at<}\DataTypeTok{uchar}\NormalTok{>(i,j) = (intensity_cdf[input.at<}\DataTypeTok{uchar}\NormalTok{>(i,j)] * max_value );}
        \NormalTok{\}}
    \NormalTok{\}}
\NormalTok{\}}
\end{Highlighting}
\end{Shaded}

Laplace sharpening function:

\begin{Shaded}
\begin{Highlighting}[]
\DataTypeTok{void}  \NormalTok{mask(Mat& input, Mat& output) \{}

    \CommentTok{// write down your code here}
    \KeywordTok{for} \NormalTok{(}\DataTypeTok{int} \NormalTok{i = }\DecValTok{0}\NormalTok{; i<output.rows; i++) \{}
        \KeywordTok{for} \NormalTok{(}\DataTypeTok{int} \NormalTok{j = }\DecValTok{0}\NormalTok{; j<output.cols; j++) \{}
            \DataTypeTok{int} \NormalTok{temp = (}\DecValTok{-4}\NormalTok{)*(input.at<}\DataTypeTok{uchar}\NormalTok{>(i, j));}
            \KeywordTok{if} \NormalTok{(i - }\DecValTok{1} \NormalTok{>= }\DecValTok{0}\NormalTok{)}
                \NormalTok{temp += input.at<}\DataTypeTok{uchar}\NormalTok{>(i - }\DecValTok{1}\NormalTok{, j);}
            \KeywordTok{if} \NormalTok{(j - }\DecValTok{1} \NormalTok{>= }\DecValTok{0}\NormalTok{)}
                \NormalTok{temp += input.at<}\DataTypeTok{uchar}\NormalTok{>(i, j - }\DecValTok{1}\NormalTok{);}
            \KeywordTok{if} \NormalTok{(i + }\DecValTok{1} \NormalTok{< output.rows)}
                \NormalTok{temp += input.at<}\DataTypeTok{uchar}\NormalTok{>(i + }\DecValTok{1}\NormalTok{, j);}
            \KeywordTok{if} \NormalTok{(j + }\DecValTok{1} \NormalTok{< output.cols)}
                \NormalTok{temp += input.at<}\DataTypeTok{uchar}\NormalTok{>(i, j + }\DecValTok{1}\NormalTok{);}
            \KeywordTok{if} \NormalTok{(temp > }\DecValTok{255}\NormalTok{)}
                \NormalTok{output.at<}\DataTypeTok{uchar}\NormalTok{>(i, j) = }\DecValTok{255}\NormalTok{;}
            \KeywordTok{else} \KeywordTok{if} \NormalTok{(temp < }\DecValTok{0}\NormalTok{)}
                \NormalTok{output.at<}\DataTypeTok{uchar}\NormalTok{>(i, j) = }\DecValTok{0}\NormalTok{;}
            \KeywordTok{else}
                \NormalTok{output.at<}\DataTypeTok{uchar}\NormalTok{>(i, j) = temp;}
        \NormalTok{\}}
    \NormalTok{\}}
\NormalTok{\}}
\end{Highlighting}
\end{Shaded}

\subparagraph{Course Content}\label{course-content-1}

CourseWeek2-1.pdf CourseWeek2-2.pdf CourseWeek2-3.pdf

\subsubsection{Lab3. Otsu threshold and Connected
component}\label{lab3.-otsu-threshold-and-connected-component}

Lab3.pdf

Otsu threshold function:

\begin{Shaded}
\begin{Highlighting}[]
\DataTypeTok{float} \NormalTok{sum = }\DecValTok{9999}\NormalTok{;}
    \DataTypeTok{int} \NormalTok{bestThreshold;}
    
    \NormalTok{vector<}\DataTypeTok{int}\NormalTok{> histo(}\DecValTok{256}\NormalTok{,}\DecValTok{0}\NormalTok{);}
    \NormalTok{findHistogram(input, histo); }\CommentTok{// a function to find histogram}
    
    
    \KeywordTok{for}\NormalTok{(}\DataTypeTok{int} \NormalTok{i=}\DecValTok{0}\NormalTok{;i<}\DecValTok{256}\NormalTok{;i++)\{}

        \NormalTok{vector<}\DataTypeTok{int}\NormalTok{> small;  }\CommentTok{// vector of smaller pixels}
        \NormalTok{vector<}\DataTypeTok{int}\NormalTok{> big;    }\CommentTok{// vector of bigger pixels}
        \KeywordTok{for}\NormalTok{(}\DataTypeTok{int} \NormalTok{j=}\DecValTok{0}\NormalTok{;j<i;j++)\{}
            \KeywordTok{for}\NormalTok{(}\DataTypeTok{int} \NormalTok{k=}\DecValTok{0}\NormalTok{;k<histo[j];k++)}
                \NormalTok{small.push_back(j);}
        \NormalTok{\}}
        \KeywordTok{for}\NormalTok{(}\DataTypeTok{int} \NormalTok{j=i;j<}\DecValTok{256}\NormalTok{;j++)\{}
            \KeywordTok{for}\NormalTok{(}\DataTypeTok{int} \NormalTok{k=}\DecValTok{0}\NormalTok{;k<histo[j];k++)}
                \NormalTok{big.push_back(j);}
        \NormalTok{\}}

        \DataTypeTok{float} \NormalTok{averageS = average(small);}
        \DataTypeTok{float} \NormalTok{averageB = average(big);}

        
        \DataTypeTok{float} \NormalTok{newSum = small.size() * variance(small, averageS, small.size()) + big.size() * variance(big, averageB, big.size());}
        \KeywordTok{if} \NormalTok{(sum == }\DecValTok{9999}\NormalTok{)\{}
            \NormalTok{sum = newSum;}
            \NormalTok{bestThreshold = i;}
        \NormalTok{\}}
        \KeywordTok{else} \KeywordTok{if}\NormalTok{(newSum <= sum)\{}
            \NormalTok{sum = newSum;}
            \NormalTok{bestThreshold = i;}
        \NormalTok{\}}
    \NormalTok{\}}

    \KeywordTok{for}\NormalTok{(}\DataTypeTok{int} \NormalTok{i=}\DecValTok{0}\NormalTok{;i<input.rows;i++)\{}
        \KeywordTok{for}\NormalTok{(}\DataTypeTok{int} \NormalTok{j=}\DecValTok{0}\NormalTok{;j<input.cols;j++)\{}
            \KeywordTok{if}\NormalTok{(input.at<}\DataTypeTok{uchar}\NormalTok{>(i,j) < bestThreshold)}
                \NormalTok{output.at<}\DataTypeTok{uchar}\NormalTok{>(i,j) = }\DecValTok{0}\NormalTok{;}
            \KeywordTok{else}
                \NormalTok{output.at<}\DataTypeTok{uchar}\NormalTok{>(i,j) = }\DecValTok{255}\NormalTok{;}
        \NormalTok{\}}
    \NormalTok{\}}

    \NormalTok{cout << }\StringTok{"Threshold: "} \NormalTok{<< bestThreshold << endl;}
\end{Highlighting}
\end{Shaded}

connected components function:

\begin{Shaded}
\begin{Highlighting}[]
\DataTypeTok{void} \NormalTok{connectedComponents(Mat& input, Mat& output)\{}
    \KeywordTok{for}\NormalTok{(}\DataTypeTok{int} \NormalTok{i=}\DecValTok{0}\NormalTok{;i<input.rows;i++)\{}
        \KeywordTok{for}\NormalTok{(}\DataTypeTok{int} \NormalTok{j=}\DecValTok{0}\NormalTok{;j<input.cols;j++)\{}
            \KeywordTok{if}\NormalTok{(input.at<}\DataTypeTok{uchar}\NormalTok{>(i,j) < }\DecValTok{200}\NormalTok{)\{}
                \NormalTok{input.at<}\DataTypeTok{uchar}\NormalTok{>(i,j) = }\DecValTok{0}\NormalTok{;}
            \NormalTok{\}}
            \KeywordTok{else} 
                \NormalTok{input.at<}\DataTypeTok{uchar}\NormalTok{>(i,j) = }\DecValTok{255}\NormalTok{;}
        \NormalTok{\}}
    \NormalTok{\}}

    \DataTypeTok{int} \NormalTok{label = }\DecValTok{50}\NormalTok{;}
    \KeywordTok{for}\NormalTok{(}\DataTypeTok{int} \NormalTok{i=}\DecValTok{0}\NormalTok{;i<input.rows;i++)\{}
        \KeywordTok{for}\NormalTok{(}\DataTypeTok{int} \NormalTok{j=}\DecValTok{0}\NormalTok{;j<input.cols;j++)\{}
            \KeywordTok{if}\NormalTok{(input.at<}\DataTypeTok{uchar}\NormalTok{>(i,j) == }\DecValTok{255}\NormalTok{)\{}
                \NormalTok{input.at<}\DataTypeTok{uchar}\NormalTok{>(i,j) = label;}
                \NormalTok{findNext(i, j, label, input.rows, input.cols, input);}
                \NormalTok{label += }\DecValTok{10}\NormalTok{;}
            \NormalTok{\}}
        \NormalTok{\}}
    \NormalTok{\}}

    \KeywordTok{for}\NormalTok{(}\DataTypeTok{int} \NormalTok{i=}\DecValTok{0}\NormalTok{;i<input.rows;i++)\{}
        \KeywordTok{for}\NormalTok{(}\DataTypeTok{int} \NormalTok{j=}\DecValTok{0}\NormalTok{;j<input.cols;j++)\{}
            \KeywordTok{if}\NormalTok{(input.at<}\DataTypeTok{uchar}\NormalTok{>(i,j) != }\DecValTok{0}\NormalTok{)\{}
                \CommentTok{//cout << input.at<uchar>(i,j) << endl;}
                \DataTypeTok{int} \NormalTok{label2 = input.at<}\DataTypeTok{uchar}\NormalTok{>(i,j);}
                \NormalTok{output.at<Vec3b>(i,j)[}\DecValTok{0}\NormalTok{] = (label2%}\DecValTok{45} \NormalTok{* }\DecValTok{531}\NormalTok{)% }\DecValTok{255}\NormalTok{;}
                \NormalTok{output.at<Vec3b>(i,j)[}\DecValTok{1}\NormalTok{] = }\DecValTok{255} \NormalTok{- label2;}
                \NormalTok{output.at<Vec3b>(i,j)[}\DecValTok{2}\NormalTok{] = (label2%}\DecValTok{30} \NormalTok{* }\DecValTok{35}\NormalTok{)% }\DecValTok{255}\NormalTok{;}
            \NormalTok{\}}
            \KeywordTok{else}\NormalTok{\{}
                \NormalTok{output.at<Vec3b>(i,j)[}\DecValTok{0}\NormalTok{] = }\DecValTok{0}\NormalTok{;}
                \NormalTok{output.at<Vec3b>(i,j)[}\DecValTok{1}\NormalTok{] = }\DecValTok{0}\NormalTok{;}
                \NormalTok{output.at<Vec3b>(i,j)[}\DecValTok{2}\NormalTok{] = }\DecValTok{0}\NormalTok{;   }
            \NormalTok{\}}
        \NormalTok{\}}
    \NormalTok{\}}
\NormalTok{\}}
\end{Highlighting}
\end{Shaded}

FindNext function to do find next unchanged pixel:

\begin{Shaded}
\begin{Highlighting}[]
\DataTypeTok{void} \NormalTok{findNext(}\DataTypeTok{int} \NormalTok{i, }\DataTypeTok{int} \NormalTok{j, }\DataTypeTok{int} \NormalTok{label, }\DataTypeTok{int} \NormalTok{rows, }\DataTypeTok{int} \NormalTok{cols, Mat& input)\{}
    \KeywordTok{if}\NormalTok{(i}\DecValTok{-1} \NormalTok{>= }\DecValTok{0}\NormalTok{)\{}
        \KeywordTok{if}\NormalTok{(input.at<}\DataTypeTok{uchar}\NormalTok{>(i}\DecValTok{-1}\NormalTok{,j) == }\DecValTok{255}\NormalTok{)\{}
            \NormalTok{input.at<}\DataTypeTok{uchar}\NormalTok{>(i}\DecValTok{-1}\NormalTok{,j) = label;}
            \NormalTok{findNext(i}\DecValTok{-1}\NormalTok{, j, label, input.rows, input.cols, input);}
        \NormalTok{\}}
        \KeywordTok{if}\NormalTok{(j}\DecValTok{-1} \NormalTok{>= }\DecValTok{0}\NormalTok{)\{}
            \KeywordTok{if}\NormalTok{(input.at<}\DataTypeTok{uchar}\NormalTok{>(i}\DecValTok{-1}\NormalTok{,j}\DecValTok{-1}\NormalTok{) == }\DecValTok{255}\NormalTok{)\{}
                \NormalTok{input.at<}\DataTypeTok{uchar}\NormalTok{>(i}\DecValTok{-1}\NormalTok{,j}\DecValTok{-1}\NormalTok{) = label;}
                \NormalTok{findNext(i}\DecValTok{-1}\NormalTok{, j}\DecValTok{-1}\NormalTok{, label, input.rows, input.cols, input);}
            \NormalTok{\}}
        \NormalTok{\}}
        \KeywordTok{if}\NormalTok{(j}\DecValTok{+1} \NormalTok{< cols)\{}
            \KeywordTok{if}\NormalTok{(input.at<}\DataTypeTok{uchar}\NormalTok{>(i}\DecValTok{-1}\NormalTok{,j}\DecValTok{+1}\NormalTok{) == }\DecValTok{255}\NormalTok{)\{}
                \NormalTok{input.at<}\DataTypeTok{uchar}\NormalTok{>(i}\DecValTok{-1}\NormalTok{,j}\DecValTok{+1}\NormalTok{) = label;}
                \NormalTok{findNext(i}\DecValTok{-1}\NormalTok{, j}\DecValTok{+1}\NormalTok{, label, input.rows, input.cols, input);}
            \NormalTok{\}}
        \NormalTok{\}}
    \NormalTok{\}}
    \KeywordTok{if}\NormalTok{(i}\DecValTok{+1} \NormalTok{< rows)\{}
        \KeywordTok{if}\NormalTok{(input.at<}\DataTypeTok{uchar}\NormalTok{>(i}\DecValTok{+1}\NormalTok{,j) == }\DecValTok{255}\NormalTok{)\{}
            \NormalTok{input.at<}\DataTypeTok{uchar}\NormalTok{>(i}\DecValTok{+1}\NormalTok{,j) = label;}
            \NormalTok{findNext(i}\DecValTok{+1}\NormalTok{, j, label, input.rows, input.cols, input);}
        \NormalTok{\}}
        \KeywordTok{if}\NormalTok{(j}\DecValTok{-1} \NormalTok{>= }\DecValTok{0}\NormalTok{)\{}
            \KeywordTok{if}\NormalTok{(input.at<}\DataTypeTok{uchar}\NormalTok{>(i}\DecValTok{+1}\NormalTok{,j}\DecValTok{-1}\NormalTok{) == }\DecValTok{255}\NormalTok{)\{}
                \NormalTok{input.at<}\DataTypeTok{uchar}\NormalTok{>(i}\DecValTok{+1}\NormalTok{,j}\DecValTok{-1}\NormalTok{) = label;}
                \NormalTok{findNext(i}\DecValTok{+1}\NormalTok{, j}\DecValTok{-1}\NormalTok{, label, input.rows, input.cols, input);}
            \NormalTok{\}}
        \NormalTok{\}}
        \KeywordTok{if}\NormalTok{(j}\DecValTok{+1} \NormalTok{< cols)\{}
            \KeywordTok{if}\NormalTok{(input.at<}\DataTypeTok{uchar}\NormalTok{>(i}\DecValTok{+1}\NormalTok{,j}\DecValTok{+1}\NormalTok{) == }\DecValTok{255}\NormalTok{)\{}
                \NormalTok{input.at<}\DataTypeTok{uchar}\NormalTok{>(i}\DecValTok{+1}\NormalTok{,j}\DecValTok{+1}\NormalTok{) = label;}
                \NormalTok{findNext(i}\DecValTok{+1}\NormalTok{, j}\DecValTok{+1}\NormalTok{, label, input.rows, input.cols, input);}
            \NormalTok{\}}
        \NormalTok{\}}
    \NormalTok{\}}
    \KeywordTok{if}\NormalTok{(j}\DecValTok{-1} \NormalTok{>= }\DecValTok{0}\NormalTok{)\{}
        \KeywordTok{if}\NormalTok{(input.at<}\DataTypeTok{uchar}\NormalTok{>(i,j}\DecValTok{-1}\NormalTok{) == }\DecValTok{255}\NormalTok{)\{}
            \NormalTok{input.at<}\DataTypeTok{uchar}\NormalTok{>(i,j}\DecValTok{-1}\NormalTok{) = label;}
            \NormalTok{findNext(i, j}\DecValTok{-1}\NormalTok{, label, input.rows, input.cols, input);}
        \NormalTok{\}}
    \NormalTok{\}}
    \KeywordTok{if}\NormalTok{(j}\DecValTok{+1} \NormalTok{< cols)\{}
        \KeywordTok{if}\NormalTok{(input.at<}\DataTypeTok{uchar}\NormalTok{>(i,j}\DecValTok{+1}\NormalTok{) == }\DecValTok{255}\NormalTok{)\{}
            \NormalTok{input.at<}\DataTypeTok{uchar}\NormalTok{>(i,j}\DecValTok{+1}\NormalTok{) = label;}
            \NormalTok{findNext(i, j}\DecValTok{+1}\NormalTok{, label, input.rows, input.cols, input);}
        \NormalTok{\}}
    \NormalTok{\}}
\NormalTok{\}}
\end{Highlighting}
\end{Shaded}

\subparagraph{Course Content}\label{course-content-2}

CourseWeek3-1.pdf CourseWeek3-2.pdf CourseWeek3-3.pdf

\end{document}
